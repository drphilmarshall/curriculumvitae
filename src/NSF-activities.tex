% d)	Synergistic Activities
%
% A list of up to five examples that demonstrate the broader impact of
% the individual’s professional and scholarly activities that focuses on
% the integration and transfer of knowledge as well as its creation.
% Examples could include, among others: innovations in teaching and
% training (e.g., development of curricular materials and pedagogical
% methods); contributions to the science of learning; development and/or
% refinement of research tools; computation methodologies, and
% algorithms for problem-solving; development of databases to support
% research and education; broadening the participation of groups
% underrepresented in science, mathematics, engineering and technology;
% and service to the scientific and engineering community outside of the
% individual’s immediate organization.


\subsection*{Synergistic Activities}

\noindent
\begin{tabular}{@{}p{2.5cm}p{15.5cm}}
2015--      & Stanford Physics Graduate Lecture Course Leader, ``Statistical Methods in Astrophysics'' \\
2014--      & Organizer and Lecturer, ``Astro Hack Week'' \\
2012--      & Co-Principal Investigator, ``Space Warps'' citizen science project \\
2011        & Workshop Organiser, ``Cosmology Meets Machine Learning,'' NIPS, Granada, Spain. \\
2010--      & Outreach Coordinator, University of Oxford Astrophysics Group and KIPAC
\end{tabular}
